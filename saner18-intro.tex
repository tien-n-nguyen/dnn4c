\section{Introduction}
\label{intro}

%\subsection{Research Problem}

%Source code is not written randomly. It is repetitive and even has
%higher regularity than natural-language texts~\cite{natural}. Based on
%that, 
%In recent years, s


Several researchers have applied statistical natural
language processing (NLP) methods to support a wide range of software
engineering (SE) tasks. For example, they leverage the capability of
predicting the next tokens of the NLP models to recommend the next
code element -- code completion~\cite{fse13,natural}, or the next API
call~\cite{ethz-pldi14,icse15} in an editing session in an
IDE. Exploring the statistical machine translation techniques in NLP,
researchers have applied them to migrate code from one programming
language to another -- language migration~\cite{ase15}, to synthesize
code from English queries -- code
synthesis~\cite{Raghothaman-ICSE16,tarlow14}, and to summarize code in
texts -- code summarization~\cite{bimodal15,rigby-icse15}.
Other applications include code convention
suggestion~\cite{barr-codeconvention-fse14}, model
testing~\cite{tonella-icse14}, and analysis of program
properties~\cite{raychev-popl15}.

%Examples of those tasks include code completion~\cite{fse13,natural},
%API recommendation~\cite{ethz-pldi14}, code
%synthesis~\cite{wei-icse16,tarlow14}, code convention
%suggestion~\cite{barr-codeconvention-fse14}, code
%summarization~\cite{bimodal15,rigby-icse15}, model
%testing~\cite{tonella-icse14}, language migration~\cite{ase15},
%program analysis~\cite{raychev-popl15}, etc. 

%The common philosophy of those approaches is that source code is
%repetitive~\cite{natural}. The repetitiveness and regularity in source
%code was leveraged to support those SE tasks.

% such as computing the likelihood of the next code token in code
% suggestion, code completion, and API
% recommendation~\cite{natural,fse13,ethz-pldi14}, computing the most
% likely code sequences in the target language for code migration or
% code synthesis~\cite{ase15,wei-icse16}, predicting the most likely
% type annotations in a program~\cite{raychev-popl15}, computing the
% likelihood of deriving feasible test cases from a
% model~\cite{tonella-icse14}, etc.

%{\bf N-gram} model is one of the most popular statistical
%LMs~\cite{manning99,ngram-wiki,dnnbook}. It assumes that a sequence of
%words is generated from left to right and the probability of a word
%being generated in that sequence is dependent only on its local
%context, i.e., a window of $n$ previously generated words. Despite the
%well-known weakness in taking only a window of prior words as the
%history, $n$-gram LM is still a state-of-the-art model. It has been
%successfully used with different abstractions to detect fine-grained
%programming patterns in source code to support the SE
%applications~\cite{natural,fse13,ethz-pldi14,tonella-icse14,ase15}.

One important denominator across those approaches 
%and applications is the concept of 
is {\em statistical language model} (LM),
%A statistical LM
which is aimed to compute a probability distribution over all possible
strings of words in a language. Given a vocabulary $V$ of words as
basic units, a language model $L$ aims to compute $P(s|L)$, which is
the probability that a sequence $s$ of the words in $V$ is
``generated'' by the model following a generative process in
$L$. To estimate such probability facilitates several
algorithms in SE applications.
%~\cite{white-msr15}.

%$N$-gram model is one of the most popular statistical
%LMs~\cite{manning99,dnnbook}. It assumes that a sequence of
%words is generated from left to right and the probability of a word
%being generated in that sequence is dependent only on its local
%context, \ie a window of $n$ previously generated words.

A recent advance in statistical language model is neural network
language model (NNLM). 
%
%NNLM has shown success in both perplexity and word error rate compared
%to conventional $n$-gram LM~\cite{DNNLM12}.  The key advantage of
%NNLMs over the counting-based $n$-gram LM is that history is no longer
%seen as exact sequences of prior words, but rather as a projection of
%the entire history into some lower dimensional
%space~\cite{dnnbook}. 
%Inspired by the recent success of Deep Neural Network (DNN) in several
%areas, 
Mikolov {\em et al.}~\cite{mikolov11,mikolov10}, Le {\em et
  al.}~\cite{le-taslp13}, and Arisoy {\em et al.}~\cite{DNNLM12} have
proposed DNN-based language models (DNN LMs) that
%. Deep neural networks
%with multiple layers
%and efficient computation process 
have been shown to capture higher-level discriminative information
about input features with multiple layers and to be able to learn
higher-level, abstract representations of the input~\cite{dnnbook}.
%NNLM has shown improvements in both perplexity and word error rate
%compared to conventional $n$-gram LM~\cite{DNNLM12}.

%Among other popular statistical LMs, neural network language models
%(NNLMs) have shown success in both perplexity and word error rate
%compared to conventional $n$-gram language models~\cite{DNNLM12}.  The
%key advantage of NNLMs over the counting-based $n$-gram LMs is that
%history is no longer seen as exact sequence of $n$-1 words, but rather
%as a projection of the entire history into some lower dimensional
%space~\cite{dnnbook}. Inspired by the recent success of Deep Neural
%Network (DNN) in several areas, Mikolov {\em et
%  al.}~\cite{mikolov11,mikolov10}, Le {\em et al.}~\cite{le-taslp13},
%and Arisoy {\em et al.}~\cite{DNNLM12} have proposed DNN-based
%language models (DNN LMs). Deep neural networks (DNNs) with more
%layers and efficient computation process have been shown to capture
%higher-level discriminative information about input features and can
%learn higher-level, abstract representations of the
%input~\cite{dnnbook}.

%The main advantage of NNLMs over the traditional counting-based N-gram
%LMs is that history is no longer seen as exact sequence of N-1 words,
%but rather as a projection of the entire history into some lower
%dimensional space. This leads to a reduction of the total number of
%parameters in the model that have to be trained, resulting in
%automatic clustering of similar histories. Compared with the
%class-based N-gram LMs, the NNLMs are different in that they project
%all words into the same low dimensional space, in which there can be
%many degrees of similarity between words. On the other hand, NNLMs
%have much larger computational complexity than N-gram LMs.

%The basic idea is to learn to associate each word in the dictionary
%with a continuous-valued vector representation, which in the
%literature is called a word embedding, where each word corresponds to
%a point in a feature space. One can imagine that each dimension of
%that space corresponds to a semantic or grammatical characteristic of
%words.

%The neural network learns to map that sequence of feature vectors to
%the probability distribution over the next word in the sequence. The
%distributed representation approach to LMs has the advantage that it
%allows the model to generalize well to sequences that are not in the
%set of training word sequences, but that are similar in terms of their
%features, i.e., their distributed representation.  Because neural
%networks tend to map nearby inputs to nearby outputs, the predictions
%corresponding to word sequences with similar features are mapped to
%similar predictions.

%+ DNN LM: (IBM and Microsoft papers)

%Neural network languagemodels (NNLMs) have shown success in both
%peplexity and word error rate (WER) compared to conventional n-gram
%language models. Most NNLMs are trained with one hidden layer. Deep
%neural networks (DNNs) with more hidden layers have been shown to
%capture higher-level discriminative information about input features,
%and thus produce better networks.

%Motivated by the success of DNNs in acoustic modeling, we explore deep
%neural network language models (DNN LMs) in this paper. Results on a
%Wall Street Journal (WSJ) task demonstrate that DNN LMs offer
%improvements over a single hidden layer NNLM.  Furthermore, our
%preliminary results are competitive with a model M language model,
%considered to be one of the current state-of-the-art techniques for
%language modeling.

%the aforementioned SE

Applying DNN LM to source code can improve accuracy in those SE
applications. Toward that goal, 
%White {\em et al.}
a few approaches~\cite{white-msr15} have directly applied the
recurrent NN-based LM on textual code tokens.  It treats code tokens
simply as words.  However, the model has difficulties in dealing with
the ambiguities in the names of program elements and API elements
(classes, method calls, and field accesses).  For example, the model
could mistakenly detect the following two different occurrences as
belonging to the same pattern: \code{x.next} for {\em accessing to the
  field} \code{next} {\em of a \code{LinkedList}} as opposed to
\code{x.next} for {\em calling to the method} \code{next} {\em of a
  \code{Scanner}} object. Moreover, the occurrences with different
variables' names, which should be considered as the same pattern, are
not recognized so.  For example, both \code{l=s.length()} and
\code{len=str.length()} are two instances of the pattern ``{\em get
  the length of a string variable and assign to an
  integer}''. However, the model cannot detect them as the same
pattern. Therefore, it cannot learn from one place and recommend for
another place.

%\subsection{Our Solution}

We conjecture that different features and contexts need
to be extracted because source code has different characteristics than
natural language texts.
%
Specifically, a program has well-defined syntax and semantics
according to the programming languages. The context regarding the
program's syntax and semantics is useful in predicting the next code
token, thus improving the estimation of occurrence likelihood of a
code sequence. 
%If a model can make use of the current syntactic unit as a context, it
%could potentially improve the prediction accuracy for the next code
%token. 
For example, in the portion of code: \code{`for (int i = 0; i < n;
  i++'}, if a model recognizes the {\em current syntactic context} of
an {\em update expression} of a \code{for} loop, it would likely
suggest `)' as the next token.
%
%Those code tokens could be far apart and cannot be captured within
%$n$-grams with reasonable sizes. 
However, if \code{i++} is a part of sequential statements, the token
\code{`;'} is likely to be suggested. In addition, the syntactic rules
in a programming language enforce the presence of pairs of code tokens
(\eg \code{try/catch}, \code{do/while}). Thus, if a model encounters
the first token, it could likely suggest the second one.
%
%Tien ---- removing ---- 
%Importantly, unlike in natural languages, a program has well-defined
%semantics, which is a valuable source for such prediction. For
%example, with data type information, a model could avoid mistakenly
%detecting \code{x.next} from \code{LinkedList.next} and \code{x.next}
%from \code{Scanner.next} as the instances of the same pattern.
%Moreover, with type information, the related APIs in a usage pattern
%(\eg \code{Iterator.next} often comes before \code{Iterator.hasNext})
%can be captured. 
%
%Importantly, if the token type (\eg variable, method call) is also
%considered, patterns at higher abstraction levels can be
%detected~\cite{fse13}, \eg \code{l=s.length()} and
%\code{len=str.length()}.
Moreover, in source code, the names of the variables, fields,
methods, classes, and API elements (method calls, field accesses)
could be ambiguous. We expect that type information is useful for a
model to distinguish them as explained earlier.

%we listed ambiguous cases on syntactic units within different
%contexts; variables, fields, classes, methods�� names; and fields and
%API method calls (e.g., the field ��next�� of LinkedList and method
%��next�� of Scanner). Contexts are useful in those cases.



%LOOK into SLAMC examples
%Semantic info: data types and token types ...  (SLAMC: suggests this
%kind of context but do not recommend at the token level. Instead, it
%recommends what the next data type or token type).

%may be not needed
%In brief, the contexual information regarding a program's syntax and
%semantics need to be considered in a model. It is not straightforward
%to decide what features should be extracted and how to integrate them
%into an DNNN LM.

In this paper, we propose to treat source code as source code (rather
than texts) by changing both the model and the input
features. Specifically, we propose to incorporate {\em syntax and type
  contexts} in a program in predicting the next code token. We also
present {\tool}, a multi-prototype DNN language model for integrating
syntax and type contexts.
%
%to adapt DNN language model for source code.  We present {\tool}, a
%DNN LM~that incorporates syntactic and type contexts for higher
%accuracy. 
%
For syntactic context, we associate code tokens with syntactic
annotations, called {\em syntaxemes}. For example, the \code{while}
loop \code{`while (i < 9) i++;'} is encoded into the syntaxeme
sequence \code{`WHILE OP EXPR CP EXPR SC'}. For type context, we
associate code~with the annotations, called {\em
  sememes}~\cite{fse13}, including {\em data types} (\eg
\code{LinkedList}) and {\em token types} (\ie field and method
declarations, field accesses, method calls, and variables). For
example, \code{length} in \code{`str.length()'} is associated with the
sememe sequence \code{CALL[String,length,0,int]} (class name:
\code{String}, method name: \code{length}, no parameter, and return
type: \code{int}). Syntax and type contexts are extracted and put
in the form of syntactic and type vectors, as parts of the model's
input together with the lexical vectors.  We used a DNN-based
method~\cite{huang12} to incorporate the contexts.
%syntactic/semantic contexts in source code.

%The syntactic and semantic contexts are incorporated via

%We~observed that with contexts, related tokens at different levels are
%captured.

%For further comparison, we also extended SLAMC into a new
%Bayesian-based LM that considers both syntactic and semantic contexts
%(note that Bayesian-based SLAMC does not include syntactic
%context). Our empirical results show that {\tool} outperforms the
%Bayesian-based LM. This demonstrates that our integration method of
%syntactic/semantic contexts via DNN is better than the integration
%method using Bayesian Inference of SLAMC. Moreover, we found that the
%Bayesian-based LM achieves higher accuracy than SLAMC, thus, showing
%also the importance of the syntactic context.

%Our intrinsic 

We evaluated Dnn4C in two SE applications: code completion and code
migration. Our empirical evaluation on several
projects shows that {\tool} relatively improves up to 44.7\%, 27.1\%,
16.3\%, and 11.6\% top-1 accuracy in code completion over the $n$-gram
model~\cite{natural}, SLAMC~\cite{fse13}, DNN LM~\cite{DNNLM12},
recurrent NN~LM \cite{white-msr15}, respectively. We observed that
even using only lexical tokens, it has up to 15.6\% relative
improvement in top-1 accuracy than the $n$-gram model.
%
%Our result shows that both our new model with DNN~and the new features
%with syntactic and semantic contexts help improve accuracy.
%
%We also made a further comparison of the feature incorporating technique
%via DNN for syntactic/semantic contexts to the feature incorporating method
%using Bayesian Inference. 
%Our empirical result shows that {\tool} outperforms the Bayesian-based
%LM
%up to 19.1\% in top-1 accuracy.
%
%We also observed that with {\tool}, related tokens at the lexical
%level are better captured. 
%
In the second application, we used {\tool} in code migration from Java
to C\texttt{\#} and showed that it helps improve migration accuracy up
to 33.2\% syntactically and 30.3\% semantically. 
%
%In the third application, we showed that {\tool} helps improve
%accuracy in code synthesis from a given English query.

%As another application, we showed that it helps improve accuracy over
%$n$-gram LM in migrating code from Java to~C\texttt{\#} with a
%phrase-based translation model.


%we also extended SLAMC into a new Bayesian-based LM that considers
%both syntactic and semantic contexts (note that Bayesian-based SLAMC
%does not include syntactic context). Our empirical results show that
%{\tool} outperforms the Bayesian-based LM. This demonstrates that our
%integration method of syntactic/semantic contexts via DNN is better
%than the integration method using Bayesian Inference of
%SLAMC.
%
%%Moreover, we also found that the Bayesian-based LM achieves higher
%%accuracy than SLAMC, thus, showing also the importance of the
%%syntactic context.
%
%For comparison, we also develop a language model that is based on
%Bayesian Inference (BI) with the integration of syntactic and semantic
%contexts into the prediction of the next lexical token. Our results
%show that {\tool} achieves ?-?\% higher accuracy than our BI LM
%model.
In this paper, our key contributions include

\begin{enumerate}

\item {\tool}, a {\em DNN language model for source code} with the
incorporation of {\em both syntactic and type contexts},

%2. a Bayesian-based language model with such integration,

%2. The code suggestion engine that is based on the model,

\item Two applications of {\tool} in {\em code completion} and {\em
  code migration}, and

\item An extensive empirical evaluation on {\tool}'s accuracy in each of
the above applications.

\end{enumerate}

%The model learns word representations that better capture the
%semantics of words, while still keeping syntactic information. These
%improved representations can be used to represent contexts for
%clustering word instances, which is used in the multi-prototype
%version of our model that accounts for words with multiple senses.

%Second, we integrate the contexts. We have two strategies:
%1) with DNN LM: {\tool}.
%2) with Bayesian Network. : compare with Bayesian model
%We found that {\tool} achieves the best.



