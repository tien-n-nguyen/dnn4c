\subsection{Accuracy With Different Contexts}

%  Table generated by Excel2LaTeX from sheet 'WithWithout'
\begin{table}[t]
  \centering
  \footnotesize
%  \tabcolsep 2pt
%  \small
  \caption{Accuracy With Different Contexts}
%(Lex+Syn+Sem={\tool})}
    \begin{tabular}{l|r|r|r|r|r|r}
    \hline
          & Top-1     & Top-2  & Top-3 & Top-4   & Top-5     & Top-10 \\
    \hline
    \code{Lex}         & 39.3\% & 53.4\% & 63.0\% & 67.2\% & 70.2\% & 77.2\%\\
    \code{Lex+Syn}     & 45.8\% & 59.7\% & 68.0\% & 72.0\% & 75.4\% & 82.5\%\\
    \code{Lex+Sem}     & 46.3\% & 61.5\% & 68.5\% & 72.5\% & 76.4\% & 82.6\%\\
    \code{{\bf Lex+Syn+Sem}} & {\bf 49.3}\% & {\bf 62.3}\% & {\bf 70.0}\% & {\bf 74.5}\% & {\bf 77.6}\% & {\bf 83.4}\%\\
    \hline
    \end{tabular}%
  \label{contextaccuracy}%
\end{table}%


%  Table generated by Excel2LaTeX from sheet 'WithWithout'
%\begin{table}[t]
%  \centering
%  \tabcolsep 2pt
%  \caption{Accuracy With Different Contexts}
%    \begin{tabular}{l|r|r|r|r|r|r}
%    \hline
%          & Top-1     & Top-2  & Top-3 & Top-4   & Top-5     & Top-10 \\
%    \hline
%    \code{Lex}         & 39.3\% & 53.4\% & 63.0\% & 67.2\% & 70.2\% & 77.2\%\\
%    \code{Lex+Syn}     & 45.8\% & 59.7\% & 68.0\% & 72.0\% & 75.4\% & 82.5\%\\
%    \code{Lex+Sem}     & 46.3\% & 61.5\% & 68.5\% & 72.5\% & 76.4\% & 82.6\%\\
%    \code{Lex+Syn+Sem} & 49.2\% & 62.3\% & 70.0\% & 74.5\% & 77.6\% & 83.4\%\\
%    \hline
%    \end{tabular}%
%  \label{contextaccuracy}%
%\end{table}%


In this experiment, we varied different context components in our
model and measured accuracy of newly configured
models. Table~\ref{contextaccuracy} shows the result. The first row is
for {\tool}~configured with only lexemes. This corresponds to the DNN
LM model in~\cite{DNNLM12}, but operates on lexemes. The second row is
for the model with both lexemes and syntaxemes. The third row is for
the model with both lexemes and sememes. The last row corresponds to
{\tool} model with all three types of features.
%We set the context's size $n$=4.

As seen, the models with contexts achieve better accuracy than the DNN
LM that treats source code as textual tokens and does not consider
syntactic and type contexts.

{\bf {\tool} with all three levels achieves even higher
accuracy} (last row). In comparison to the state-of-the-art DNN LM
(operating on lexemes), {\tool} has good relative improvement in
top-1 accuracy: 25.2\% (10\% absolutely).
%16.3\% (top-1) and 6.1\% (top-5). 
%
Importantly, it achieves high accuracy. In one out of two cases,
with a single suggestion, {\tool} correctly suggests the
next token. 
%In 3 out of 4 cases, the correct token is in the list of 4
%suggestions from {\tool}}. 
With~5 suggestions, it suggests the correct token in 77.6\% of the
cases.
%
%
With the addition of only syntaxemes, {\em the relative improvement in
top-1 accuracy (i.e., with a single suggestion) is 16.5\% (6.5\%
absolute improvement)}. 

%There is an interesting observation.  Despite that the context size
%$n$ is only 4 (which might not contain all syntaxemes of a syntactic
%unit, e.g., a \code{for} loop), {\tool} is still able to capture the
%beginning of a syntactic unit. For example, the syntaxemes \code{FOR}
%or \code{FORINIT} indicate the start of a \code{for} statement. The
%model uses it to sufficiently help predict the next token within the
%context defined by the unit.

%
%8.27%
%Examining the cases, we can see that with the syntaxemes as syntactic
%context, the lexical tokens relevant to surrounding ones are ranked
%higher because the grammar rules have restricted the valid syntactic
%units at a suggestion point. 
%Concrete examples are presented in Section~\ref{cases}.
%
%Moreover, 

Combining type context via sememes with lexemes improves
better than adding syntactic context via syntaxemes to lexemes
(i.e.,
\code{Lex+Sem} $>$ \code{Lex+Syn}). The model \code{Lex+Sem}
relatively improves 18.1\% at
%and 1.5\%
top-1 accuracy over the model \code{Lex}.
% and \code{Lex+Syn}, respectively.
%
After investigating, we found that data types and token types allow
\code{Lex+Sem} to rank the correct token at a suggestion point
higher than \code{Lex+Syn} with the syntactic context of surrounding
syntactic units. For example, pairs of API calls that often go
together (e.g., \code{Scanner.hasNext} and \code{Scanner.next}) are a
good foundation to suggest the second one if the first one is
encountered. In this case, \code{Lex+Syn} ranks multiple
identifiers (for method calls) higher than other types of
tokens, but \code{Scanner.next} is not the top one.

%After investigating, we found that data types and token types limit
%more the potentially valid tokens at a suggestion point than the
%syntactic context of current and containing syntactic units. In some
%suggestion cases, with syntactic context, the list of candidate tokens
%is in the tens of tokens. However, the sememes with data and token
%types can shorten that list to a few candidates. For example, pairs of
%method calls that often go together are a good indication for
%suggesting the second one if the first one is encountered.





