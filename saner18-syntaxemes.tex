\subsection{Syntactic Context}
\label{syntaxsec}

%To extract a syntactic information for the current code, we build a sequence
%of special syntax symbols, called {\bf syntaxemes}, for the sequence
%of code tokens in the current file. The current syntactic context
%consists of $n$-1 syntaxemes prior to the current location.
%Syntaxemes are the basic units of syntax that represent the
%(non)terminal symbols on the right hand side of grammar rules of a
%programming language. For example, let us consider the following:

\input{Table_javastruct-2}

\noindent {\bf Overview.} To represent the syntactic context in source
code, we encode a source file with a sequence of special syntactic
symbols, called {\bf syntaxemes}~\cite{ase15}.
% which was introduced in the work
%mppSMT~\cite{ase15}. 
They represent the basic units of syntax that represent the (non)-terminal
symbols on the right hand sides of grammar rules of a programming
language. The rationale is that such (non)terminal symbols of the
grammar rules represent the syntactic structures~in~a program. For
example, the code \code{`for (int i = 0; i < n; i++) count++;'} is
encoded into the syntaxeme sequence \code{FOR OP INIT SC EXPR SC
  UPDATE CP EXPR SC} (will be explained later). The current syntactic
context consists of $n$ syntaxemes prior to the current code location.

%Following mppSMT~\cite{ase15}, 
To build the syntaxeme sequence, we traverse the parse tree of a
program, produce and ensemble the syntaxemes corresponding to all
(non)terminals. For efficiency, we want to represent only
coarse-grained syntactic structures and do not go further than
\code{Expression}.  For example, two expressions \code{i < n} and
\code{count++} in the above example are not broken down further than
\code{EXPR} to collect syntaxemes.

\vspace{0.05in}
\noindent {\bf Details.}  Let us present the function {\bf
  \code{encode}} that we use to recursively produce the sequence of
syntaxemes for source code. Table~\ref{tab:javastruct} shows an
excerpt from~\cite{ase15} of the rules for different Java syntactic
units to produce the corresponding syntaxeme sequences, which are
listed on the right window.
% (the full details can be found in~\cite{ase15}). 
First, we parse the code to build the parse tree.
%the important encoding rules for different Java syntactic units to
%produce the syntaxeme sequences (which are listed on the right
%panel). Encoding rules for others are similar.
If the code is not parseable, we use the partial program analyzer 
PPA~\cite{ppa08} to create the parse tree.
%, and then we produce the syntaxemes from it. 
For the unparseable tokens, their lexical tokens are kept and
annotated with the special syntaxeme \code{LEX}.
%
%Let us explain how we produce syntaxemes for a method from its parse
%tree.

From the parse tree of a method, we produce the syntaxeme sequence as
follows.
%
We traverse the parse tree to match the method declaration and the
code in the body to {\em recursively} produce the corresponding
sequences of syntaxemes.  To explore the fine-grained syntactic units,
the non-terminal symbols (\eg~\code{Statement}, \code{StmtList},
\code{Block}, \code{CaseSection}, \code{Body}, etc.) in the syntactic
rules are recursively converted into syntaxemes until expressions are
encountered or there is no more such non-terminal.
%
On the right hand side of Table~\ref{tab:javastruct}, the capital
letters are the produced syntaxemes and not expanded further,
and the parts that are called with \code{encode} will be expanded.
%
The resulting syntaxemes at all the steps are recursively
concatenated to create a larger sequence. Finally, the syntaxeme sequence is used
as a syntactic context.
%
Let us take an example.

\begin{lstlisting}[basicstyle=\scriptsize\sffamily, stepnumber=1, numbers=left, language=Java, aboveskip=1pt,  belowskip=1pt, numbersep=-5pt]
  public int PrintAll(ArrayList a) {
    for (i = 0; i < a.length(); i++)
      System.println(a.get(i));
  }
\end{lstlisting}
We encode this method declaration using the rules in
Table~\ref{tab:javastruct}:\\ \fbox{\code{MOD TYPE ID OP {\bf encode}(ParamList) CP {\bf encode}(Block)}}
%
where \code{OP} and \code{CP} are the terminal symbols for the
separators in the grammar; \code{MOD}, \code{TYPE}, and \code{ID}
represent the modifier \code{public}, the type \code{int}, and the
identifier \code{PrintAll}.
%where \code{MOD}, \code{OP}, \code{CP}, \code{OB}, \code{SC}, and
%\code{CB} are the modifier, and the terminal symbols in the grammar,
%\code{ID} representing the identifier \code{ClientQueryResult} and the
%separators.
%
\code{PrintAll} is the lexeme of that \code{ID}, representing
the method's name.
%
\code{ParamList} and \code{Block} are expanded further via other
rules. For example, \code{Block} is encoded into `\code{OB \code{{\bf
      encode}(For)} CB}', which becomes \code{OB FOR OP INIT SC EXPR SC
  UPDATE CP EXPR SC CB} after {\bf encode}(For) is run. Then, \code{a.length()} and
\code{System.println(...)}  are not expanded further since we
use sememes for them (Section~\ref{sememesec}).
%
\code{ParamList} is for the parameter list and is expanded into the
sequence `\code{PARAM}' for `\code{ArrayList a}'. We do not expand
a parameter since we use sememe for it.
%(Section~\ref{sememesec}).

%We will parse that method declaration as follows:\\
%\fbox{\code{Modifiers ID OP ParamList CP OB {\bf SuperCall} SC CB}} (1)

%where \code{ID}, \code{OP}, \code{CP}, \code{OB}, \code{SC}, and
%\code{CB} are terminal symbols in the grammar, representing the
%identifier \code{ClientQueryResult} and the separators.
%\code{ClientQueryResult} is the lexeme of that \code{ID}, representing
%the method's name. In contrast, \code{Modifiers}, \code{ParamList},
%and \code{SuperCall} are non-terminal symbols, which are
%encoded/expanded further via other rules. For example,
%\code{Modifiers} is for \code{public}. \code{ParamList} is for the
%parameter list and is expanded into the sequence `\code{ID ID CO ID
%  ID}' for `\code{Transaction ta, int initialSize}'. \code{SuperCall}
%is encoded into `\code{\textbf{super} OP ParamList CP}', in which
%\code{\textbf{super}} is a keyword, and so on.
%==================================================

%%(denoted with bold typeface in lowercase text)
%and \code{ParamList} is for the parameters.

%If the code is not parseable, we use a partial parser such as
%PPA~\cite{ppa08} to create an AST and then we produce the syntaxemes
%from the AST. For the unparseable tokens, their lexical tokens are
%kept and annotated with the special syntaxeme \code{LEX}.


%For example, the statement \code{for (int i=0; i$<$10; i++)
%  \{System.out.\-println(i);\}} in Java matches the encoding rule for
%a \code{for} statement, thus, its syntaxeme sequence is the
%concatenation of the sequence \code{FOR OP INIT SC EXPR SC UPDATE CP}
%and the syntaxeme sequence for the block statement
%\code{\{System.out.println(i);\}}. The latter sequence is produced
%recursively by the function \code{encode} for a \code{Block} as
%\code{OB EXPR SC CB} since \code{System.out.println(i)} is a
%StatementExpression.

%(a rule for \code{Block} was used).

%We will continue to encode at the deeper levels until there is no more
%symbol \code{Statement}. That is, the capital letters in the right
%panel of Table~\ref{tab:javastruct} are the produced syntaxemes and
%not expanded further.
%Finally, such syntaxemes are recursively
%concatenated to~create~a~larger sequence. Syntaxeme sequences will be
%used as syntactic~contexts.



