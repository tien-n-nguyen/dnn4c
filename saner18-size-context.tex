\subsubsection{Accuracy with Different Sizes of Contexts}



We conducted an experiment to study the impact of the size~$n$ of the
contexts on {\tool}'s accuracy. Table~\ref{nsizeaccuracy} shows
results for different values of $n$. As seen, when increasing the size
of the context window (for all three levels of lexemes, syntaxemes,
and sememes) from 2 to 4, the accuracy increases since more contexts
are captured for suggestion (especially, accuracy increases much for
$n$ from 2 to 3). However, {\em when $n=4$--$7$, the
accuracy~is~stable}. When $n > $ 7, the number of sequences is
extremely large for DNN, causing scalability problem.  This suggests
us to use $n$=4 in other experiments.
%
This result is consistent with the finding in $n$-gram for texts in
NLP in which $n$-grams for $n$=3--6 give best
performance~\cite{jurafsky14}.
%

Another interesting observation is that despite that the context size
$n$ is only 4 (which might not contain all syntaxemes of a syntactic
unit, \eg a \code{for} loop), {\tool} is still able to capture well
the current syntactic unit because each syntaxeme allows {\tool} to
represent multiple lexical tokens.
%the beginning of a syntactic unit. 
For example, the syntaxemes \code{FOR} or \code{FORINIT} indicate the
start of a \code{for} statement. The model uses it to sufficiently
help predict the next token within the context defined by the~unit.

% Table generated by Excel2LaTeX from sheet 'sensitivity_n-gram'
\begin{table}[t]
  \centering
%  \tabcolsep 3pt
  %\small
  \caption{Accuracy With Different Sizes of Contexts}
%(Lex+Syn+Sem={\tool})}
    \begin{tabular}{l|r|r|r|r|r|r}
    \hline
     n     & Top-1 & Top-2 & Top-3 & Top-4 & Top-5 & Top-10\\
    \hline
    2 & 31.0\% & 44.8\% & 54.3\% & 59.2\% & 63.4\% & 77.2\% \\
    3 & 46.1\% & 61.2\% & 68.4\% & 73.5\% & 76.4\% & 83.3\% \\
    4 & 49.2\% & 62.3\% & 70.0\% & 74.5\% & 77.6\% & 83.4\% \\
    5 & 49.1\% & 62.2\% & 69.9\% & 74.4\% & 77.4\% & 83.3\% \\
    6 & 49.1\% & 62.1\% & 69.8\% & 74.2\% & 77.4\% & 83.3\% \\
    7 & 49.1\% & 62.1\% & 69.8\% & 74.2\% & 77.4\% & 83.3\% \\
    \hline
     \end{tabular}%
  \label{nsizeaccuracy}%
\end{table}%

%\begin{table}[t]
%  \centering
%  \tabcolsep 3pt
%  \caption{Accuracy With Different Sizes of Contexts}
%    \begin{tabular}{l|r|r|r|r|r|r|r}
%    \hline
%     n     & Top-1 & Top-2 & Top-3 & Top-4 & Top-5 & Top-10 & Top-20 \\
%    \hline
%    2 & 31.0\% & 44.8\% & 54.3\% & 59.2\% & 63.4\% & 77.2\% & 83.5\% \\
%    3 & 46.1\% & 61.2\% & 68.4\% & 73.5\% & 76.4\% & 83.3\% & 85.6\% \\
%    4 & 49.2\% & 62.3\% & 70.0\% & 74.5\% & 77.6\% & 83.4\% & 85.6\% \\
%   5 & 49.1\% & 62.2\% & 69.9\% & 74.4\% & 77.4\% & 83.3\% & 85.6\% \\
%    6 & 49.1\% & 62.1\% & 69.8\% & 74.2\% & 77.4\% & 83.3\% & 85.6\% \\
%    \hline
%    \end{tabular}%
%  \label{nsizeaccuracy}%
%\end{table}%


For $n$=4, we also measured top-1 accuracy for the second token of the
pairs of the syntactic tokens (\eg \code{do}
and \code{while}, \code{try} and \code{catch},
\code{\{ and \}}, $($ and $)$, etc.) and a few pairs of popular API elements
(\eg \code{Iterator.hasNext}/\code{Iterator.next},
\code{Enumeration.hasMoreElements/Enumeration\-.nextElement}, and
\code{StringTokenizer\-.hasMoreTokens/StringTokenizer.\-nextToken}).  We
found that with even such a small window of contexts, {\tool} is able
to capture the associations between pairs of related, yet distant code
tokens with a top-1 accuracy from 62--75\%. 




%This is expected because DNN LM with context~\cite{huang12} has that
%capability for natural-language texts. This is important for a
%SMT-based code migration engine to produce syntactically correct
%code.

