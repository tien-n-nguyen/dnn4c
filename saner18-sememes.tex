\subsection{Type Context}
\label{sememesec}

%\input{semantic}

%\input{sememe}

%\subsection{Integrating Semantic Information}

%The success of SMT relies much on the ability of a model to learn the
%mappings between respective source code in both languages, and apply
%to translate for the new code.

%Because source code has well-defined semantics, {\tool} needs to
%capture/learn the meaningful pieces of code and their corresponding
%code in the target language at a high level of abstraction. Currently,
%we encode and integrate local semantic information.

%We aim to integrate semantic information into SMT 

%To capture/learn the mappings of meaningful pieces of code including
%API mappings and the mappings of concrete names of variables/fields,
%we encode semantic information to the source code.

To capture the type context at a fine-grained level, we encode
such information to the code tokens.
%
To do that, for a sequence of code tokens, we build a sequence of
type annotations, called {\bf sememes}, which are the symbols
representing type information. We adopted sememes from an existing
work, SLAMC \cite{fse13}.
%
%\begin{Definition}[Sememe]
{\em The \emph{sememe} of a code token at a code location is a
  structured annotation representing its type information,
  including its {\bf token type/role} and {\bf data type}}~\cite{fse13}.
%\end{Definition}

\begin{table}[t]
  \centering
  \footnotesize
%  \tabcolsep 6pt
  \caption{Examples of Sememes~\cite{fse13}}
  \vspace{-0.06in}
    \begin{tabular}{llll}%{l@{}l}
    \addlinespace
    \toprule
    & {\bf Token} & \textbf{Token Role} & \textbf{Sememe} \\
    \midrule
    1 & \code{Iterator} & data type &  TYPE[Iterator]  \\
    2 & \code{iter} & variable   &   VAR[Iterator]   \\
    3 & \code{``Vancouver''} & literal   &  LIT[String] \\
%    MethDecl  & \code{indexOf} $\rightarrow$ FUNC[String,indexOf, \\
%              &    \quad \quad \quad \quad \quad \quad \quad  TYPE[String] PARA[String],int]  \\
    4 & \code{println} & method call  &  CALL[System,println,1,String,void]  \\
    5 & \code{a} & parameter   &  PARA[ArrayList]   \\
    6 & \code{length} & field access    & FIELDACC[String[],length]  \\
    7 & = & operator &  OP[assign]  \\
    8 & null & special literal   & NULL\\
    \bottomrule
    \end{tabular}%
  \label{tab:stoken}%
\end{table}%

%
Specifically, {\tool} first encodes the data types of code tokens if
they are available. If semantic analysis is not successful, lexical
tokens are kept and annotated with the special sememe \code{LEX}.
%Tien
%%With that, it could avoid mistakenly considering code of different
%%purposes as the same.
%e.g., \code{x.next} for accessing to the field \code{next} of a
%\code{LinkedList} as oppose to \code{x.next} for calling to the method
%\code{next} of a \code{Scanner}.
%The {\em data type} information is recorded for variables, fields, and
%literals. For a method, its signature including its name, enclosing
%class name, return type, parameters and their types are recorded.
% e.g., whether they are {\em field} and {\em method
%  declarations, field accesses, method calls, variables}, etc. 
%
Second, {\tool} records the token types. Token
type refers to the role of a token in a program.~For example, we 
encode whether it is a variable, a literal, a keyword, an operator, a
method call, a method declaration, a field, a field access, a type, or
a class.
%Each code token has a role in a program according to the written
%programming language, e.g., whether it is a {\em type, variable,
%  literal, operator, keyword, method call, function declaration,
%  field}, or {\em class}. 
For example, in \code{``a.length()''}, after semantic analysis,
\code{a} is recognized as a token with its role of a variable, while
the role of \code{length} is a method call.
%
%For example, \code{java.io.File.exists()} in Java and
%\code{System.IO.File.Exists(file)} in C\texttt{\#} are the
%corresponding calls. With token roles, we hope to align them since
%they are used~to {\em ``check the existence of a file given a file
%  name''} in Java and C\texttt{\#}.  
The return types and parameters' types in a method call are also
captured. 

%Data types help capture frequent API usages. For example, a call to
%\code{Scanner.hasNext} is often followed by a call to
%\code{Scanner.next}. That would help better migration of API elements
%between two languages as shown in~\cite{ase15}.



%---- old ---
%\begin{table}[t]
%  \centering
%  \small
%  \caption{Examples of Sememes~\cite{fse13}}
%    \begin{tabular}{ll}%{l@{}l}
%    \addlinespace
%    \toprule
%    \textbf{Token Role} & \textbf{Token $\rightarrow$ Sememe} \\
%    \midrule
%    Data type & \code{LinkedList} $\rightarrow$ TYPE[LinkedList]  \\
%    Variable   & \code{str}  $\rightarrow$ VAR[String]   \\
%    Literal    & \code{``Hello World!''} $\rightarrow$ LIT[String] \\
%    MethDecl  & \code{indexOf} $\rightarrow$ FUNC[String,indexOf, \\
%              &    \quad \quad \quad \quad \quad \quad \quad  TYPE[String] PARA[String],int]  \\
%    MethCall  & \code{exists} $\rightarrow$ CALL[File,exists,0,boolean]  \\
%    ParamName     & \code{filename}  $\rightarrow$ PARA[String]   \\
%    FieldAcc       & \code{next} $\rightarrow$ FIELD[LinkedList,next]  \\
%    Operator      & = $\rightarrow$ OP[assign], . $\rightarrow$ OP[access]  \\
%    Special literal   & 0 $\rightarrow$ ZERO, \code{``''} $\rightarrow$ EMPTY, \code{null} $\rightarrow$ NULL\\
%    \bottomrule
%    \end{tabular}%
%  \label{tab:stoken}%
%\end{table}%
%------------------


Let us show some examples of sememes for the popular types of tokens
in Table~\ref{tab:stoken}. If the type resolution for data types,
variables, and literals is successful, the sememe sequence with the
type annotation is produced for them. For example, the first three
rows of Table~\ref{tab:stoken}, the model keeps the data type with the
annotations \code{TYPE} (data type), \code{VAR} (variable), and
\code{LIT} (literal). The sememes for a variable and a literal do not
include their lexemes since they are stored at the lexeme
sequence. For a method call, its class name (\code{System}), its name
(\code{println}), the number of passing parameter (1), the parameter's
type (\code{String}), and the return type (\code{void}) are kept.  That
sememe sequence represents the semantic annotation of \code{println}.

%Table~\ref{tab:stoken} shows the examples of popular types of sememes.
%For example, in the code \code{File.exists()}, \code{exists} is a
%function call and its sememe consists of the symbols `\code{CALL}',
%`[', its class name \code{File}, its name \code{exists}, the number of
%  passing parameter (0), the return type \code{boolean}, and ']'. That
%sememe sequence represents the semantic value of that token.  As
%another example, the sememe sequence for
%\code{current.getEdge().setMarked(true)} is \code{VAR[Rectangle] CALL
%  [Rectangle,getEdge,0,Edge] PE CALL [Edge,setMarked,1,boolean,void]}.

%Let us take an example of different styles in Java and C\texttt{\#}.
%A pair of method calls becomes two field accesses and an
%assignment: \code{current.getEdge().setMarked(true)} $\rightarrow$
%\code{current.edge.marked = true}. The Java sememe sequence\\
%\code{VARREF[Rectangle] CALL [Rectangle,getEdge,0,null,Edge] CALL
%  [Edge, setMarked,1,boolean,void]} becomes\\
%\code{VARREF[Rectangle] FIELD [Rectangle, edge] FIELD [Edge, marked] ASSIGN LIT[boolean]} in C\texttt{\#}.

%{\tool} was able to map and translate those two sememe sequences for a
%project during our empirical study (Section~IX).

%Table 2
%A field access is modeled similarly to a method call.
We also encode special values with special sememes, \eg 
%a
%\code{EMPTY} for an empty string, \code{ZERO} and 
~\code{NULL} for \code{null}, since they could be used for special
meaning, \eg successful execution or nullity checking. The
separators and keywords are marked with
sememes that are the same as their syntaxemes. Two sememes differ by
one field are considered as distinct.



%The rationale is that those special values are usually used to
%indicate special meaning such as a successful execution, nullity
%checking in a program, etc.

%%%The separator tokens, e.g. semicolons and parentheses, and keywords
%%%are not associated with semantic information, thus are excluded in
%%%sememes and handled at the lexical level. If semantic information is
%%%not available, the lexical token is kept and annotated with the sememe
%%%of type \code{LEX}. The sememe for a variable and that for a literal
%%%do not include their lexemes because they are handled at the lexical
%%%level. {\tool} encodes special values with special sememes such as
%%%\code{NULL} for the \code{null} value, \code{EMPTY} for the empty
%%%string, \code{ZERO} for the zero value. The rationale is that those
%%%special values are usually used to indicate special meaning such as a
%%%successful execution or nullity checking in a program.
